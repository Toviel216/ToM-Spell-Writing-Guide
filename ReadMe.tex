

%%%%%%%%%%%%%%%%%%%%%%%%%%%%%%%%%%%%%%%%%%%%%%%%%%%%%%%%%%%%%%%%%%%%%
This is a document that lets you draw spells with just a few commands.

Overview:
    Commands.tex :      Look up the names i gave the commands.
    Coordinates.tex :   Modify the coordinates to your needs
    Main.tex :          Add your Commands here
    Preamble.tex :      Necessary code that lets you make drawings
    ReadMe.tex :        You are reading it rn

How to use:
    1. Look up the command you want to use in Commands.tex
    2. Add the command in main.tex / replace the ones given
    3. Compile

The predefined coordinates are a polygon with eleven edges. The coordinate (1) is to the right.

Notes:
    - There are two definitions of the connections: All starting at 1 ("normal"), and the symmetric ones from the Spell Writing Guide. They are the same commands, but with a SWG at the end
    - If you want to use custom coordinates, put a % in front of the lower part in coordinates.tex and remove the % from the upper part. Now change the definitions
    - Select main.tex to compile, every other file could yield an error
    - remove the % at the lowest part of coordinates.tex to show small dots where the edges are. Change the number in the circle (number) to pick the radius
    - Check out my Guide (in the discord) to find more ways to customize your result
    - The link only lets you read the files. Import them to edit and make your own spellbook. Change the documentclass to "article" (in Preamble.tex) and start writing
    